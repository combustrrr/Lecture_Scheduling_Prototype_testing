% Abstract
\chapter*{Abstract}
\thispagestyle{empty}
\addcontentsline{toc}{chapter}{Abstract}

The \textbf{Timetable Buddy} is a comprehensive lecture scheduling system designed to streamline the process of managing academic schedules for educational institutions. The system addresses the challenges faced by students, faculty, and administrators in coordinating class schedules, managing enrollments, and accessing timetable information efficiently.

This web-based application provides a centralized platform for managing lecture slots, course enrollments, and student-faculty interactions. The system features role-based access control, supporting three distinct user types: administrators, faculty members, and students. Each role has specific functionalities tailored to their needs.

Key features include real-time lecture slot management, automated enrollment processing with waitlist capabilities, conflict detection for overlapping schedules, and comprehensive dashboard views for different user roles. The system is built using modern web technologies including React for the frontend, Node.js with Express for the backend, and MongoDB for data persistence.

The project follows industry-standard software development practices, including comprehensive test planning, risk assessment, and quality assurance procedures. A complete test case planning and execution document has been developed, covering 60 test cases across various functional areas. Additionally, a risk assessment framework has been implemented to identify and mitigate potential project risks.

The system aims to improve the efficiency of academic schedule management, reduce scheduling conflicts, and enhance the overall user experience for all stakeholders in the educational ecosystem.

\textbf{Keywords:} Lecture Scheduling, Timetable Management, Educational Technology, Web Application, Enrollment System, MERN Stack
